\section{Přednáška 5 -- Rozpoznání kompilátoru}

Rozpoznání kompilátoru umožňuje zmenšit množství analyzovaného kódu -- známé výstupy kompilátoru mohou být vynechány.
Rozpoznáním jsme schopni určit rozložení dat a kódu uvnitř binárního souboru.

Kompilátory od MS lze poznat pomocí linkovaných DDL a speciálního dekorování (začínají znakem \texttt{?}).

Z kódu lze typicky vynechat např. vstupní bod programu (je silně závislý na kompilátoru, nepřináší mnoho informací).
Stejně tak není nutné analyzovat knihovní kód -- je možné využít dokumentaci (?).

\subsection*{Signatury knihoven}

Knihovny se rozpoznávají pomocí signatur.
Signatury se vytváří na základě zdrojového kódu.
Signaturu každé funkce vytvoříme např. pomocí prvních \texttt{X} bytů strojového kódu funkce.

\subsection*{Rozpoznání linkovaných knihoven}

Dynamicky linkované knihovny je možné rozpoznat pomocí programů, které prochází importní tabulku.
Pro staticky linkované knihovny je možné hledat informace (o verzích, copyrightu, chybových hlášení, \dots) v uložených řetězcích.
