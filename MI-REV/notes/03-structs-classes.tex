\section{Přednáška 3 -- Struktury a třídy}

Struktury a třídy jsou v C++ ekvivalentní.
Jediným rozdílem jsou přístupové modifikátory -- \texttt{struct} používá jako výchozí veřejné rozhraní, \texttt{class} používá privátní.

Jednotlivé členské proměnné funkce jsou vždy uloženy za sebou s použitím zarovnání definovaným ABI.

\subsection*{Alokace struktur}

Alokovat můžeme např. pomocí \texttt{malloc}, pak jako návratovou hodnotu dostaneme a manuálně je následně potřeba k ukazateli doalokovat proměnné.

Při konstrukci pomocí operátoru \texttt{new} je volání podobné, místo manuálního přiřazení proměnných se volá konstruktor (ukazatel na \texttt{this} je předán pomocí registru/parametru).

\subsection*{Dědičnost}

Při dědičnosti objektu je v paměti rozložen objekt tak, že nejdříve jsou vloženy nadtřídy až pak samotná třída.

Obsahuje-li datový typ i virtuální metody, \textbf{musí} také obsahovat virtuální tabulku metod (VMT).
Ukazatel \texttt{this} ukazuje na VMT (pokud existuje).
Všechny metody jsou pak volány výhradně přes VMT.
Uspořádání v paměti: VMT, data, VMT2, data2, \dots

Třída má VMT i pokud je abstraktní (a obsahuje virtuální metody).
VMT je přiřazována v konstruktoru.

Díky VMT můžeme zjistit některé informace:

\begin{itemize}
    \item použít ukazatel na VMT ke zjištění, jestli je objekt nějakého typu,
    \item nalézt vícenasobnou dědičnost,
    \item pomocí VMT nalézt jaký kód patří k danému objektu.
\end{itemize}

\subsection*{Complete Object Locator}

Nezdokumentovaná struktura nacházející se před VMT.
Obsahuje (mimo jiné) \texttt{TypeDescriptor}, který má v sobě ukazatel na VMT, a \textit{mangled} název type -- lze použít k ověření dědičnosti.
Dále obsahuje \texttt{ClassHierarchyDescriptor}, ta obsahuje počet \textit{baseclass} a pole \texttt{BaseClassArray}, kde každý prvek je \texttt{BaseClassDescriptor} (obsahuje podobné informace jako COL).

Díky COL se dá zjistit jméno třídy a hierarchie tříd.
Je ale nutné mít zapnutou podporu RTTI.
