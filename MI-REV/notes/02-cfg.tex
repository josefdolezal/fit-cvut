\section{Přednáška 2 -- CFG}

Znázorně tok kódu pomocí grafu.
Graf dělíme do (základní) bloků -- maximální posloupnosti po sobě jdoucích instrukcí, do které lze vstoupit jedině první a opustit jedině poslední instrukcí.

\subsection*{Podmínka -- If}

\img{cfg-if.png}{Podmínka if}{cfg-if}{0.6}

\subsection*{Podmínka -- If-then-else}

\img{cfg-if-else.png}{Podmínka if-then-else}{cfg-if-else}{0.8}

\subsection*{Cyklus -- While}

\img{cfg-while.png}{Cyklus while}{cfg-while}{0.8}

\subsection*{Cyklus -- For}

\img{cfg-for.png}{Cyklus for}{cfg-for}{0.7}

Pokud vynecháme přikaz inkrementování, dostaneme podobnou strukturu jako u \texttt{while}.
Při dekompilaci není jednoznačně možné určit jaký typ smyčky byl použit.

\subsection*{Switch}

\img{cfg-switch-1.png}{Switch pomocí sub/dec}{cfg-switch-1}{0.8}
\img{cfg-switch-2.png}{Switch pomocí cmp}{cfg-switch-2}{0.8}
\img{cfg-switch-3.png}{Switch pomocí skokové tabulky}{cfg-switch-3}{0.8}

\subsection*{Vstupní bod}

Vstupní bod není metoda \texttt{main}, ale jiná -- inicializační -- funkce.
Její RVA je uvedena v poli \texttt{AddressOfEntryPoint}.
Poskytuje jí runtime knihovna, může se jmenovat např. \texttt{mainCRTStartup} (závisí např. na použití Unicodu).

Volá funkce \texttt{initterm\_e} (volající inicializaci zařízenou kompilátorem) a \texttt{initterm} (připravující argumenty pro main, volání konstruktorů a globálních proměnných).
Následuje volání main funkce programu.

Pro volání funkci na konci programu (např. destrukce objektů) můžeme zaregistrovat handler metodou \texttt{atexit}, která při ukončení programu (funkcí \texttt{exit}) postupně tyto funkce volá.

\subsection*{Kódování ukazatelů}

Ukazatele na zásobníku by mohly být útočníkem přepsány ke spuštění jiného kódu.
Lze tomu zabránit pomocí API funkcí \texttt{EncodePointer} a \texttt{DecodePointer}.

\subsection*{Hot Patching}

Při zapnutné podpoře Hot Patching mají před sebou funkce 5 volných \texttt{nop} instrukcí.
Funkce navíc začínají \uv{prázdnou} instrukcí \texttt{mov edi, edi}.
To dává dohromady 7B, které je možné využít ke krátkému skoku na žátek \texttt{nop} sekce a následně využít klasický skok na novou funkci.
To bez potřeby restartu.

\subsection*{Nepřímé skoky}

Umožňují za běhu nahradit libovolnou za běhu.
Nahrazenou funkci můžeme samostatně zkompilovat a \uv{injektovat} do běžící funkce.

\subsection*{Import Address Table}

Umístěna v sekci \texttt{.rdata} (je tedy pouze pro čtení).
Obsahuje nalinkované moduly a jejich jména symbolů jsou v importním adresáři (\texttt{ImportDirectory}).
Všechna volání externího API prochází jediným bodem programu -- záznamem v IAT.

IAT je standardně v sekci pouze pro čtení -- pro modifikace (není standardní chování) je potřeba blok paměti \uv{odemknout}.
