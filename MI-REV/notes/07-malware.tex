\section{Přednáška 7 -- Malware}

Malware (malicious software -- škodlivý software) je software vytvoření za účelem poškození uživatele počítače.

Malware existuje spousta typů:

\begin{itemize}
    \item špatně napsaná aplikace,
    \item nechtěná aplikace,
    \item poškozená aplikace,
    \item data,
    \item nechtěná komunikace,
    \item škodlivý email.
\end{itemize}

\subsection*{Potensionálně nechtěný software}

PUA nebo PUP (potentially unwanted program) je program, který sám o sobě nepůsobí škodu ale je distribuován bez povolení uživatele nebo pomocí sociálního inženýství.
Případně to může být aplikace, kterou útočníci často zneužívají.
Rozlišujeme více kategorií PUA:

\begin{description}
    \item[Unsafe] Aplikace zneužívaná malwarem,
    \item[Unwanted] aplikace známá tím, že je instalována bez vědomí uživatele,
    \item[Suspicious] má podobné chování jako malware ale není známá.  
\end{description}

\subsection*{Klasifikace malware}

\begin{description}
    \item[Virus] Infikuje binárku tak, že její funkčnost nezmění nijak kromě přidaného spuštění viru -- otevírá a mění různé soubory na disku, schovává se do kódu nebo entry pointu,
    \item[Worm] kopíruje se na další systémy (pomocí USB, P2P, Mail, \dots) -- na USB si vytvoří samospouštěcí sekvenci,
    \item[Trojan] vše ostatní.
\end{description}

\subsection*{Klasifikace Trojan}

\begin{description}
    \item[Backdoor] Umožňuje spouštět vzdálené příkazy,
    \item[Adware] upravuje nebo vynucuje zobrazovanou reklamu,
    \item[Spy] kontinuálně krade data,
    \item[Downloader] menší malware, který je použitý ke stažení jiného,
    \item[Exploit] zranitelnost v systému,
    \item[Rootkit] malware napadající jádro systému, brání uživatelské viry,
    \item[Bootkit] infikuje master boot record,
    \item[Ransomeware] blokuje funkcionality PC a požaduje výkupné.
\end{description}

\subsection*{Složení viru}

Každý vir má typicky následující složení:

\begin{itemize}
    \item Downloader,
    \item Kryptik -- zabaluje (pack) kód,
    \item Injector -- spouští škodlivý kód v jiném procesu,
    \item Malware
\end{itemize}

\subsection*{Persistence}

Priorita malware je zůstat v systému co nejdéle.
Může se maskovat za dynamicky linkovanou knihovnu a systém jí následně slinkuje.

Často se také maskuje/injektuje do běžných procesů (svchost.exe, explorer.exe, csrss.exe).

\subsection*{Obrana viru}

\begin{itemize}
    \item Dva procesy, které navzájem kontrolují existenci -- jakmile je jeden zabit, druhý ho ihned spustí znovu,
    \item proces otevře svůj vlastní soubor aby nebylo možné ho smazat/přesunout,
    \item připojuje k sobě debugger,
    \item hook na API mazání souborů,
    \item injektáž do více procesů.
\end{itemize}

\subsection*{Monetizace}

Malware může monetizovat několika způsoby -- využitím zdrojů (botnet, miner), kradení osobních informací (heslo do banky, \dots), výhružky (ransom) a inzerce (výměna inzerce, přidání inzerce, falešná inzerce).
