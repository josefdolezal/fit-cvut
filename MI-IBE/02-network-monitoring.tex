\section{Monitorování sítě}

Jak již bylo zmíněno, behaviorální analýza je nadstavbou nad monitorováním datových toků.
Monitorování datových toků meziúrovní monitorování počítačových sítí.
Pod ní, na nejnižší úrovni, se nachází úroveň jednoduchého monitoringu pomocí Simple Network Management Protokolu (zkr. \textit{SNMP}).
Na nejvyšší úrovnik se naopak nachází tzv. paketová analýza.

\subsection{Simple Network Management Protocol}

Protokol \texttt{SNMP} slouží k potřebám správy sítí a sběru dat.
Sbíraná data jsou následně podrobována vyhodnocování.
Porotokol rozlišuje dvě strany: monitorovaný (hlídaný systém) a monitorovací (sběrna dat).
Na monitorovaném systému je nasazen \texttt{agent} (modul komunikující s monitorovacím systémem), který reaguje na požadavky monitorovacího systému nebo mu odesílá informace o významných událostech (\textit{SNMP Trap}).

Na této úrovni se typicky zkoumá počet přenesených dat a počet paketů, které k tomu byly využity.
Na základě nasbíraných informací umožňuje tento protokol vytvořit jednoduchý přehled využití sítě ale i zařízení v ní (např. využití CPU, RAM a disku nebo informace o selhání zařízení).
Tento přístup ovšem neumožňuje získat detailní přehled o síti.

\subsection{Monitorování datových toků}

Chceme-li získat detailní přehled o dění v síti, musíme přistoupit k monitorování datových toků.
Tok je definován jako množina IP paketů procházejících skrz monitorovací službu v určitý interval.
Měření probíhájí na základě IP toků, kde se analyzují pouze hlavičky paketů (obsah paketů není analyzován ani uchováván).

Tato technika je implementována například otevřeným protokolem NetFlow od společnosti Cisco.

\img{flow-monitoring.png}{Monitorování datových toků}{flow-monitoring}

Obrázek \ref{fig:flow-monitoring} zobrazuje průběh zachytávání paketů.
Výsledkem analýzy toku paketů je:

\begin{itemize}
    \item zdrojová a cílová IP adresa,
    \item zdrojový a cílový port,
    \item protokol,
    \item počet přenesných bajtů a paketů,
    \item délka toku.
\end{itemize}

Zdrojem dat mohou být směrovače a přepínače, koncová zařízení nebo např. firewally.
Nasbíraná data se ukládají do tzv. kolektorů.
Ty následně umožňují k výsledkům přistupovat v reálném čase nebo k záznamům v historii.

Tato technika pomáhá k efektivnímu řešení problémů v síti a ke zvýšení bezpečnosti.
Lze díky ní rychle odhalit vnitřní (např. \textit{malware}), ale i vnější (např. \textit{DDOS}) útoky.

Kromě bezpečnostního přehledu sítě je možné monitorování využít k dohledu nad využíváním internetu.
Pomocí toků lze například zjistit, jestli uživatelé v síti uměle nenavyšují zátěž přistupováním k obsahu, který v danný okamžit nesouvisí s jejich náplní práce.

Dalším využitím může být sledování dodržování \textit{Service-level agreement} dostupnosti sítě od dodavatele konektivity.

V neposlední řadě je možné odhalit také špatnou konfiguraci aplikací uvnitř sítě.
Ze síťového provozu lze totiž odhalit nadměrné využití sítě některou z aplikací či její připojování na části sítě, ke kterým by neměla mít přístup.

\subsection{Paketová analýza}

Kromě analýzy samotných hlaviček paketů je možné zkoumat také jejich obsah.
Tato technika se nazývá \textit{paketová analýza}.
Odchytávání paketů se provádí pomocí \textit{síťových analyzátorů} -- to může být počítačový program, ale i specializovaný hardware.
Analyzátor zachytává pakety v celé síti nebo její podsíti.

\subsubsection*{Nevýhody paketové analýzy}
Přesto, že paketová analýza může poskytnout detailnější informace o síťovém provozu, má oproti monitorování datových toků několik nevýhod.

První nevýhodou je časová náročnost.
Dekódovat spolu s hlavičkami i obsah paketů je časově i paměťově náročnější a tedy se neprovádí nepřetržitě.
Analýza se typicky provádí pouze v konkrétní čas na konkrétním vzorku komunikace.

Druhou nevýhodou může být riziko právních problémů.
Komunikace může obsahovat osobní údaje, které bez souhlasu uživatele není možné zpracovávat a tedy je nelze zkoumat ani při síťové analýze.
