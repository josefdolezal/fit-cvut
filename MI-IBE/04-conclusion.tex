\section{Závěr}

Behaviorální analýza sítě je důležitým prvkem síťové bezpečnosti.
Díky sledování anomálií oproti výchozímu chování sítě je možné odhalovat nejen standardní hrozby odhalitelné i nástroji jako firewall nebo antimalware, ale i pokročilé hrozby (APT) a útoky typu zero-day.

NBA je velmi účinný na hledání hrozeb, měl by ale být vždy používán jako doplněk ke standardním řešením.
Koncové stanice by měly být samostatně chráněné pomocí antimalware, síť pak pomocí firewall a systémů IDS a IPS.

Analýza je prováděna pouze nad hlavičkami paketů, nehrozí tedy únik nebo zneužití citlivých údajů obsažených v těle paketů.
I tak je ale možné (na úrovni IP) sledovat mezi kým komunikace probíhá a na základě takových informací uživatele sítě kompromitovat.
Je tedy důležité před nasazením této technologie i taková rizika zvážit.

Přestože díky moderní architektuře nasazení je pořizovací cena dosažitelná i pro menší podniky, typické využití této technologie je u firem s větší síťovou infrastrukturou.
To je dáno mimo jiné tím, že přínos pro menší podniky nemusí odpovídat pořizovacím nákladům a také tím, že tyto firmy nemají ve svém týmu odborníky, kteří by výsledky analýzy vyhodnocovali.

Největší rozmach tato technologie měla pravděpodobně mezi lety 2007 až 2013.
Z této doby také existuje většina dostupné literatury, která navíc předpovídá slibnou budoucnost NBA.
Od roku 2010 ale zmínky o této technologii utichají.
Bez hlubšího zkoumání nelze jednoznačně říct, zda je technologie stále využívána nebo byla v průběhu let nahrazena modernějšími přístupy.

Osobně hodnotím tuto technologii velmi pozitivně.
Její přínos bezpečnosti díky možnosti odhalit zero-day útoky je nezanedbatelný a firmy by měly zvážit využití NBA jako doplňku stávajících zabezpečení.
