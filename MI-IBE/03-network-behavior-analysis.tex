\section{Behaviorální analýza síťového provozu}

V dnešní době je poměrně běžné nasazení technik proti standardním hrozbám v síti -- firewall, web filtr nebo vzdálený přístup.
Spolu s těmito technikami jsou v síti chráněny i samostatná koncová zařízení pomocí antivirů, personálních firewallů nebo antimalweru.

Taková ochrana již ale není dostačující.
Objevují se pokročilé hrozby (\textit{Advanced Persistent Threats}, zkr. \texttt{APT}) nebo \textit{zero-day útoky}, které narozdíl od standardních hrozeb nejsou jednoduše detekovatelné.

\subsection{Advanced Persisten Threats}

Útoky tohoto typu se od standardních útoků vyznačují tím, že necílí na širokou skupinu uživatelů.
Jsou tedy vysoce sofistikované a typicky jsou prováděné skupinou odborníků.

Tyto útoky jsou rozdělené do několika fází.
V první fázi útočník sleduje aktivity uživatele (případně skupiny uživatelů).
Pomocí sociálního inženýrství se snaží z uživatele vylákat informace o přístupu k síti a na tu následně zútočit.

Po průniku do sítě se snaží útočník rozšířit škodlivý kód a získat přístupová oprávnění do většího množství koncových stanic či datových uložišť a serverů.
Šíření kódu probíhá pomalu a opatrně za využití málo známých bezpečnostních chyb.
Po rozšíření sbírá škodlivý software citlivá data z vnitřku sítě, která opatrně vynáší.
Aby útok nebyl snadno rozpoznatelný při vynášení dat (např. pomocí datových toků nebo paketové analýzy), jsou tato data šifrována a ukryta v gigabitech běžného provozu.

Takové útoky jsou vysoce sofistikované a přizpůsobené pro konkrétní případy.
Z tohoto důvodu je pro bezpečnostní programy (antivir, antimalware) složité takový typ útoku odhalit.
Antiviry totiž pro vyhledávání škodlivého kódu využívají tzv. \texttt{signatury}
\footnote{
    Při analyzování škodlivého kódu antivirové společnosti extrahují tzv. signaturu viru, která ho jednoznačně identifikuje.
    Tato signatura je následně uložena do databáze signatur, kterou antivirus používá při hledání hrozeb na koncové stanici.
    Tato metoda ale není dokonalá, protože často není schopna odhalit tzv. polymorfní viry -- tedy viry, které za běhu mění části svého kódu a tím i svou signaturu.
}.
Pro sofistikované útoky ale typicky signatury nejsou v čase provádění útoku dostupné a tak je standardní bezpečnostní řešení nemohou rozpoznat.

\subsection{Zero day útok}

Mezi další těžko rozpoznatelné útoky patří kromě \texttt{APT} také tzv. \textit{Zero Day útoky}.
Tyto útoky a hrozby se snaží využívat zranitelností v software (případně hardware, viz Meltdown a Spectre), které nejsou obecně známé nebo pro ně neexistuje obrana (např. aktualizace).

Název vychází z podstaty zranitelnosti -- tedy že uživatel je výchozím zranitelném stavu (\textit{nultém dni}) až do doby, kdy je vydána oprava.
V tomto stavu mohou být uživatelé dny, měsíce ale i roky v závislosti na rychlosti distribuce opravy.

Stejně jako u APT ani k těmto útokům neexistuje do doby odhalení žádná signatura a jsou tedy pro bezpečnostní programy obvykle neviditelné.

\subsection{Analýza chování sítě}

Analýza chování sítě (\textit{Network Behavior Analysis}, zkr. \texttt{NBA}) navazuje na ostatní bezpečnostní řešení tam, kde jejich možnosti rozpoznání hrozeb končí.
Dnešní podnikové sítě typicky obsahují ochrany jako je vynucování bezpečnostních politik pomocí \textit{firewallu} nebo systémy \texttt{IDS}\footnote{\texttt{IDS} -- Systém pro odhalení průniku do sítě. Na základě monitorování provozu detekuje aktivity vedoucí k průniku do sítě a zasahuje proti nim.} a \texttt{IPS}\footnote{\texttt{IPS} -- Systém prevence průniku. Jedná se o rozšíření systému \texttt{IDS}, oproti kterému je bezpečnostní zařízení instalováno in-line (v síťové cestě) díky čemuž může nebezpečí. aktivně předcházet.} automaticky blokující nežádoucí provoz.
Ani kombinace těchto metod ale nedokáže blokovat veškeré možné nebezpečí.

Těmto případům se \texttt{NBA} snaží předcházet detekováním anomálií v síťovém provozu.

\subsubsection*{Princip NBA}

Protože \texttt{NBA} hledá anomálie oproti standardnímu chování, je nejprve nutné toto chování definovat.
Standardní chování určíme vychozím provozem v síti.
Jinak řečeno, v první fázi zaznamenáváme chování uživatelů a zařízení v síti za běžných podmínek a vytváříme tzv. \textit{profil chování}.
Během vytváření výchozího profilu si musíme být jisti, že provoz sítě opravdu odpovídá reálnému provozu, a že nejsou prováděny potencionálně nebezpečné operace (v opačném případě hrozí, že již existující hrozby nebudou nikdy odhaleny).
Nástroje \texttt{NBA} následně detekují anomálie v chování, které by u běžných bezpečnostních opatření mohly zůstat bez povšimnutí.
Útoko na síť mohou probíhat hodiny až dny, tato metoda navíc umožňuje určit, jaké všechny systémy byly napadeny (toto firewall ani systémy \texttt{IDS} ani \texttt{IPS} sami nezvládnou).

Kromě samotné detekce tyto nástroje navíc anomálie seskupují a zmenšují tak množství informací, které bezpečnostní analytici musí zkoumat.
U velkých společností \texttt{NBA} běžně detekuje mezi dvěma až třemi tisíci anomálií denně.
Ty jsou ale seskupeny podle vzájemného propojení nebo vzoru chování.
Analytici tak nemusí zkoumat nepřeberné množství incidentů a mohou se na útoky soutředit jako na celek.

\subsubsection*{Nasazení NBA}

Jak již bylo zmíněno, \texttt{NBA} je nadstavou nad analýzou síťových toků.
Nasazení je tedy úzce spojeno s nasazením měření síťových toků.
Jednou z možností je využití otevřeného protokolu NetFlow společnosti Cisco, který se stal standardem pro tuto problematiku.

Architektura NetFlow se skládá z několika exporterů\footnote{NetFlow exportér -- zařízení zapojené na lince analyzující procházející pakety. Na základě IP toků generuje NetFlow statistiky, jejichž obsah je vidět na obrázku \ref{fig:flow-monitoring}.} a jednoho kolektoru\footnote{NetFlow kolektor -- zařízení s velkokapacitním uložištěm, které sbírá statistiky z exporterů a dlouhodobě je uchovává. Nad těmito daty následně probíhá analýza chování.}.
Při nasazování rozlišujeme dva typy architektur: tradiční a moderní.

\subsubsection*{Tradiční architektura}

Tato architektura podle společnosti Cisco předpokládá na pozici exportérů směrovače.
Ty tedy kromě svého standardního určení navíc počítají NetFlow statistiky.
Tato architektura má však hned několik nevýhod.

První nevýhodou je vysoká pořizovací cena.
Standardní směrovače nepodporují funkce exportéru a je tedy nutné využít speciálních směrovačů, které jsou ale oproti standardním výrazně dražší.

Další nevýhodou je snížení výkonu směrovače, který navíc musí počítat statistiky a odesílat je do kolektoru.
To lze vyřešit snížením počtu analyzovaných paketů (např. pomocí vzorkování každého $n$-tého paketu).
Vzrokováním ale snižujeme účinnost analytických nástrojů a tim i pravděpodobnost odhalení hrozby.
Návrh této architektury je vidět na obrázku \ref{fig:netflow-standard-arch}.

\img{netflow-standard-arch}{NetFlow -- Tradiční architektura}{netflow-standard-arch}

\subsubsection*{Moderní architektura}

Nedostatky tradiční architektury řeší moderní přidáním pasivní NetFlow sondy.
NetFlow sonda je zařízení, specializující se na monitorování provozu a vytváření statistik.
Procházející pakety jsou pouze monitorovány a sondou nejsou nijak upraveny.
Oproti řešení tradiční architektury jsou sondy levné a narozdíl do směrovačů mohou být zapojeny do libovolného bodu v síti.

Sonda přenáší exportované statistiky do kolekturu po samostatné lince, což je pro monitorovanou linku činí neviditelnými.

Návrh moderní architektury je na obrázku \ref{fig:netflow-modern-arch}.

\img{netflow-modern-arch}{NetFlow -- Moderní architektura}{netflow-modern-arch}

\subsubsection*{Přínosy NBA}

Tato sekce shrnuje přínosy behaviorální analýzy síťového provozu oproti standardním bezpečnostním opatřením.

Díky analýze chování sítě lze na základě detekce anomálií odhalovat vnitřní i vnější útoky na síť.
Dalším přínosem je objevování známého i neznámého malware v síti i přes neexistenci signatur pro takové hrozby.
Z hlediska vnitřní firemní bezpečnosti je možné odhalovat i potencionálně nechtěné úniky citlivých dat ze společnosti.
Taková data by mohli uživatelé vynést například na cloudová uložiště nebo jiná uložiště nespadající do infrastruktury společnosti.

V neposlední řadě lze díky analýze paketů dohledávat a prokazovat provozní a bezpečnostní incidenty uvnitř společnosti.
