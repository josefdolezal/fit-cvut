\section{Limitní věty}

Cílem je studování náhodného experimentu na základě jeho opakování.

\subsubsection*{Výběrový průměr}

\begin{align*}
    \overline{X}_n     &= \frac{1}{n}\sum_{i=1}^{n}{X_i} \\
    E\overline{X}_n    &= \mu \\
    var \overline{X}_n &= \frac{\sigma^2}{n}
\end{align*}

\subsubsection*{Součet}

\begin{align*}
    S_n     &= \sum_{i=1}^{n}{X_i} \\
    ES_n    &= n\mu \\
    var S_n &= n\sigma^2
\end{align*}

\subsection{Silný zákon velkých čísel}

Nechť jsou $X_1, \dots, X_n$ i.i.d. náhodné veličiny se střední hodnotou $EX_i = \mu$, potom s rostoucím $n$ $\overline{X}_n$ konverguje k $\mu$.

\subsubsection*{Slabý zákon velkých čísel}

Narozdíl od SZVK požaduje konečný rozpty, ale místo nezávislosti veličin stačí jejich nekorelovanost.

\subsection{Centrální limitní věta}

\subsubsection*{Konvergence v distribuci}

Posloupnost $X_1, \dots$ náhodných veličin s distribučními funkcemi $F_{X_1}, \dots$ a $X$ je náhodná veličina s distribuční funkcí $F_X$.
$X_i$ konvergují k $X$ v distribuci ($X_n \rightarrow^{\mathcal{D}} X$), jestliže

$$
    \lim_{n\rightarrow\infty}{F_{X_n}(x)} = F_X(x)
$$


\subsubsection*{Centrální limitní věta}

$X_1, \dots$ jsou i.i.d. náhodných veličin s konečnou střední hodnotou $EX_i=\mu$ a konečným rozptylem $\textrm{var}X_i = \sigma^2 > 0$.
Nechť

$$
    Z_n = \frac{S_n - n\mu}{\sigma\sqrt{n}}
$$

Potom $Z_n$ \textit{konverguje v distribuci} k $N(0,1)$: $Z_n \rightarrow^{D} N(0,1)$ pro $n$ jdoucí do nekonečna.