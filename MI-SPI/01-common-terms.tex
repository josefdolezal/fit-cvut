\section{Přednáška 1 - Základní pojmy}

\begin{description}
    \item[Pravděpodobnost] matematická kvantifikace náhody
    \item[Statistika] odhad pravděpodobnosti pomocí experimentálních dat (na základě pokusů se ověřují modely) 
\end{description}

\subsection{Pravděpodobností prostor}

Pravděpodobností prostor (experiment) je základním pojmem teorie pravděpodobnosti.
Jedná se o trojici $\mathcal{E} = (\Omega, \mathcal{F}, P)$, kde

\begin{itemize}
    \item $\Omega$ je prostor elementárních jevů (zhrnuje všechny výsledky experimentu),
    \item $\mathcal{F}$ náhodné jevy -- kolekce tvořená podmnožinami $\Omega$,
    \item $P$ je pravděpodobnost přiřazovaná náhodným jevům.
\end{itemize}

Elementárním jevem nazýváme možné výsledky experimentu (jsou navzájem exkluzivní a vyčerpávající).
Množinu všech těchto jevů nazýváme prostor elementárních jevů nebo také výběrový prostor (z angl. \textit{sample space}).

Náhodný jev $A$ je množina elementárních jevů ($A \subset \Omega $), kterým můžeme přiřadit pravděpodobnost.
Ne všechny podmnožiny $\Omega$ jsou náhodné jevy (nelze jim přiřadit pravděpodobnost).

Kolekce $\mathcal{F}$ podmnožin prostoru $\Omega$ se nazývá $\sigma$-algebrou, pokud:

\begin{itemize}
    \item $\emptyset \subset \mathcal{F}$,
    \item obsahuje opačný jev každému jevu $(A \in \mathcal{F} \Rightarrow A^c \in \mathcal{F})$
    \item obsahuje spočetné sjednocení jevů ($A_1, A_2, \dots \in \mathcal{F} \Rightarrow \bigcap_{i=1}^{\infty}{A_i} \in \mathcal{F}$).
\end{itemize}

\subsection{Pravděpodobnostní míra}

Zobrazení $P$ na měřitelném prostoru $(\Omega, \mathcal{F})$, $P: \mathcal{F} \rightarrow \mathbb{R}$ splňující:

\begin{itemize}
    \item nezápornost: $\forall{A} \in \mathcal{F}: P(A) \geq 0$,
    \item normalizace: $P(\Omega) = 1$
    \item $\sigma$-aditivita: jsou-li jevy disjunktní, pak pravděpodobnost sjednocení rovna součtu pravděpodobností ($P(\bigcup_{i=1}^{\infty}{A_i}) = \sum_{i=1}{P(A_i)}$).
\end{itemize}

\subsection{Podmíněná pravděpodobnost}

$A$ a $B$ jsou náhodné jevy, kde $P(B) \geq 0$. Podmíněná pravděpodobnost $A$ za podmínky jebu $B$ je definována jako:

$$
    P(A|B) = \frac{P(A \cap B)}{P(B)}
$$  

\subsubsection*{Věta o úplném rozkladu}
$B_1, B_2, \dots, B_n$ je rozklad $\Omega$ takový, že $\forall{i}: P(B_i) \geq 0$ a $A$ je náhodný jev kde $P(A) > 0$. Potom:

$$
    P(B_j|A) = \frac{P(A|B_j)P(B_j)}{\sum_{i=1}^{n}{P(A|B_i)P(B_i)}}
$$

\subsubsection*{Nezávislot náhodných jevů}

Jevy $A$ a $B$ jsou nezávislé, pokud platí:

$$
    P(A \cap B) = P(A) \cdot P(B)
$$

Obecně platí pro $n$ jevů, že \uv{pravděpodobnost průniku je rovna pravděpodobnosti součinu}.

Jsou-li $A$ a $B$ nezávislé, pak platí:

$$
    P(A|B) = \frac{P(A \cap B)}{P(B)} = \frac{P(A)P(B)}{P(B)} = P(A)
$$

Tedy jsou-li jevy nezávislé, nedodá mi výsledek jednoho jevu žádnou informaci o tom, jak dopadne druhý.
