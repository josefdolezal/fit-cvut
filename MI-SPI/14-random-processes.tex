\section{Náhodné procesy}

Nechť \pspace je pravděpodobnostní prostor a $T\subseteq \mathbb{R}$ indexová množina.
Systém náhodných veličin

$$
    X = \{X_t | t \in T\}, X_t: \Omega \rightarrow \mathbb{R}
$$

se nazývám \textit{reálný náhodný proces}.

\begin{itemize}
    \item Množninu $T$ lze chápat jako čas, je-li nejvýše spočetná pak se jedná o diskrétní čas ${X_n | n\in \mathbb{N}_0}$,
    \item je-li nespočetná, pak se jedná o spojitý čas (např. ${X_t | t \geq 0}$),
    \item množina hnodnot $S \subseteq \mathbb{R}$, kterých nabývají veličiny $X_t$ je množina stavů (diskrétní nebo spojitá). 
\end{itemize}

\subsection{Trajektorie náhodného procesu}

Proces $X$ lze chápat jako zobrazení z $\Omega$ do prostoru funkcí

$$
    X: \Omega \rightarrow \{f: T \rightarrow S\}
$$

Hustotu procesu $X_t(\omega)$ lze chápat jako hodnotu funkce $X(t;\omega)$ (tedy jevu $\omega$ v čase $t$).

Trajektorie (realizace) procesu $X=\{X_t | t\in T\}$ kde $\omega \in \Omega$ je elementární jev, je funkce $f: T \rightarrow \mathbb{R}$

$$
    f(t) := X_t(\omega), \forall{t \in T}
$$

\subsection{Rozdělení náhodného procesu}

Systém všech rozdělení konečně rozměrných vektorů $(X_{t_1}, X_{t_2}, \dots, X_{t_n})$, tedy systém konečně rozměrných distribučních funkcí.

$$
    F_{X_{t_1},\dots}(x_1, \dots, x_k) = P(X_{t_1} \leq x_1, \dots, X_{t_k} \leq x_k)
$$

Pro spojité lze proces charakterizovat systémem sdružených hustot: $f_{X_{t_1}, \dots}(x_1, \dots)$

Pro diskrétní lze proces charakterizovat systémem sdružených pravděpodobností $P(X_{t_1} = x_1, \dots)$.

\subsection{Střední hodnota procesu}

Funkce $\mu_X: T \rightarrow \mathbb{R}$ definovanou v bodě $t \in T$ jako

$$
    \mu_X(t):= EX_t
$$

Je-li $\mu_X(t) \equiv 0$, je proces centrovaný.

\subsection{Rozptyl procesu}

Funkce $\sigma_{X}^2: T \rightarrow [0, + \infty)$ v $t \in T$:

$$
    \sigma_{X}^2 := \textrm{var} X_t = E(X_t - EX_t)^2 = EX_{t}^2 - \mu_{X}^{2}(t)
$$

\subsection{Kovariace procesu}

Autokovariační funkce $c_X(t,s): T \times T \rightarrow \mathbb{R}$:

$$
    c_X(t, s) := \textrm{cov}(X_t, X_s) = E(X_t - EX_t)(X_s-EX_s) = EX_tX_s - \mu_X(t)\mu_X(s)
$$

Autokorelační funkce je $r_X: T \times T \rightarrow [-1, 1]$:

$$
    r_X(t,s) = \varphi(X_t, X_s) = \frac{c_X(t,s)}{\sigma_X(t)\sigma_X(s)}
$$

\subsection{Stacionarita procesu}

Zkoumání změny vlastností v čase -- některé se mohou měnit.
Zkoumá se rozdělení, rozdělení při posunu v čase, střední hodnota a konvariace (závislost na $t$ a $s$ nebo jen na odstupu $t-s$).

\subsubsection*{Striktně stacionární}

Pokud jsou všechna konečně rozměrná rozdělení invariantní vůči posunu v čase -- $\forall{k \in \mathbb{N}}, \forall{t_1, \dots, t_k \in T}, \forall{h \in \mathbb{R}}: t_1+h,\dots, t_k+h \in T$ a $\forall{x_1, \dots, x_n \in \mathbb{R}}$.

$$
    F_{X_{t_1}, \dots, X_{t_k}}(x_1, \dots, x_k) = F_{X_{t_1 + h}, \dots, X_{t_k+h}}(x_1, \dots, x_k)
$$

Proces je \textit{stacionární řádu $N$}, pokud podmínka pro stacionaritu platí pro všechny $k$-tice takové, že $k \leq N$.

\subsubsection*{Slabá stacionarita}

\begin{description}
    \item[Stacionární ve stř. hodnotě:] $\mu_X(t) = \mu, \forall t \in T$,
    \item[Stacionární v autokovarianci:] pokud $c_X(t,s)$ je funkcí pouze $t-s$, pak
        $$
            c_X(t) := c_X(t, 0), \quad r_X(t) := r_X(t,0) = \frac{c_X(t)}{c_X(0)}
        $$ 
    \item[Slabě stacionární:] stacionání ve střední hodnotě i autokovarianci
\end{description}

Je-li proces striktně stacionární, pak je i slabě stacionární.
