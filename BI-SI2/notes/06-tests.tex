\section{Testování}
  \begin{itemize}
    \item Zkoušení/simulace provozu
    \item Analýza s cílem nalézt chyby SW
    \item Zhodnocení atributů schopností SW
  \end{itemize}

  \begin{description}
    \item[Test plan] Definuje strategii testů, obsahuje test coverage, test methods a responsibilities.
    \item[Test case] Množina podmínek, za kterých tester určí, zda aplikace funguje správně.
    \item[Test script] Množina instrukcí, které budou během testování provedeny.
    \item[Test data] Data speciálně identifikovaná pro využití v rámci testování.
    \item[Test report] Výsledek jednoho či více testů.
  \end{description}

  \subsection{Principy}
    \begin{itemize}
      \item Kompletní testování není možná
      \item Měření a sledování je důležité
    \end{itemize}

  \subsection{Typy testů}
    \subsubsection{Developer}
      Testování samotným vývojářem před přidáním do Unit testů.

    \subsubsection{Unit testy}
      Testy základních celků (třídy a metody).

    \subsubsection{Integrační testy}
      \begin{itemize}
        \item Komunikace mezi jednotlivými systémy.
        \item Test integrace základních celků do systému.
      \end{itemize}

    \subsubsection{Smoke testy}
      Spouštějí se v okamžiku, kdy je dokončen vývoj aplikace a lze ji spustit. Pouští se po integračních testech.

      \begin{itemize}
        \item Testuje implementaci, instalaci a spuštění jednotlivých částí systému.
        \item Zaměřuje se na hlavní části systému.
        \item Pokud jsou testy splněny, přechází se k systémovému testování.
        \item U menších projektů kde nejsou ostatní testy se provádí většinou pouze \emph{smoke testy}.
      \end{itemize}

    \subsubsection{Systémové}
      Testování aplikace jako celku. Testují se všechny části spolu s dodávaným HW. Obsahuje test nároků na HW.

    \subsubsection{Kvalifikační}
      Kontrola splnění požadavů na SW. Odehrávají se u dodavatele před dodávkou. Pokud SW projde
      je dodán zákazníkovi.

    \subsubsection{Akceptační}
      Obdoba \emph{kvalifikačních testů}, které se odehrávají u zákazníka. Provádí se podle připravených scénářů vytvořených
      ve spolupráci s dodavatelem.

    \subsubsection{Regresní testy}
      Testují, zda změny SW (např. nové funkce) nenarušily funkcionalitu stávajících částí, na které by změny
      neměly mít vliv.

  \subsection{Testovací techniky}

    \subsubsection{Black box testy}
      \begin{itemize}
        \item Testuje se oproti rozhraní
        \item Implementace není známá
        \item Není nutné často upravovat
      \end{itemize}

    \subsubsection{White box}
      \begin{itemize}
        \item Strukturované testy
        \item Přihlíží se k implementaci
        \item Změna implementaci často testy rozbije
      \end{itemize}

  \subsection{Požadavky na testy}
    \begin{itemize}
      \item Najít chyby aby byly odstraněny
      \item Rozhodnout se jestli aplikaci nasadit
      \item Minimalizovat náklady na podporu
      \item Měření kvality
    \end{itemize}
