\section{Projektové řízení}
  \begin{description}
    \item[Projekt] Dočasná organizační jednotka, která ze vytvořena s cílem doručit jeden nebo více produktů podle obchodního plánu.
  \end{description}

  \subsection{Charakteristiky projektu}
    \begin{itemize}
      \item Proměnlivý
      \item Dočasný
      \item Unikátní
      \item Nejistý
    \end{itemize}

  \subsection{Problémy IT projektů}
    \begin{itemize}
      \item Nerealistické termíny
      \item Změny v rozsahu
      \item Špatné řízení rizik
      \item Špatná komunakice
      \item Viditelnost pokroku
      \item Špatně definovaná vize a cíle projektu
    \end{itemize}

  \subsection{Aspekty projektu}
    \begin{itemize}
      \item Náklady
      \item Čas
      \item Rozsah
    \end{itemize}

  \subsection{Metodiky}

    \subsubsection{Prince2}
      Obecná metodika pro řízení projektů (nejen v IT). Stojí na sedmi principech:

      \begin{itemize}
        \item Business justification - Jak projekt nebo úkol splňují své cíle
        \item Learn from experience
        \item Role a odpovědnosti
        \item Manage by stages - Více release a iterací je jedna stage
        \item Manage by exception - Řízení zaměřené na identifikaci a řešení situací vyhýbající se normálu
        \item Zaměření na produkty
        \item Přizpůsobení prostředí projektu (velikosti, složitosti, důležitosti, ...)
      \end{itemize}

  \subsection{Projektový manažer}
    Udržuje plán projektu aktuální, stará se o:

    \begin{itemize}
      \item Termíny
      \item Závazky
      \item Zdroje
      \item Rizika
      \item Běh projektu
    \end{itemize}

    Pracuje pro lidi na projektu a reportuje stav jak své firmě, tak zákazníkovi.
    Jeho náplní je odstínit tým od nepříjemností. Zodpovídá za to, co dělají členové týmu.
    Rozděluje úkoly a kontroluje jejich splnění.

    \subsubsection{Self management}
      \begin{itemize}
        \item Je důležité zvládnout organizeci vlastního času
        \item Úkoly do jedné minuty plnit ihned
        \item Rozlišovat mezi důležitými a urgentními úkoly
      \end{itemize}

  \subsection{Nástroje projektového manažera}
    \begin{itemize}
      \item Plán projektu, WBS
        \begin{itemize}
          \item Menší úkoly na 1-5MD
          \item Měří aktuální stav projektu
          \item Sleduje earned values
        \end{itemize}
      \item Výkazy
        \begin{itemize}
          \item Hlídání odvedené práce
          \item Kontrola zbývající práce
        \end{itemize}
      \item Nabídka
        \begin{itemize}
          \item Cenotvorba a termíny
          \item Odhady a předpoklady, zdroje, milníky, harmonogram, ...
        \end{itemize}
    \end{itemize}

    \subsection{Řízení rizik}
      \begin{description}
        \item[Riziko] Ohrožení projektu, ceny, termínu, kvality nebo jiné vlastnosti projektu. Může se jednat o ohrožení \emph{Business case}.
      \end{description}

    \subsection{Měření a metriky}
      \begin{description}
        \item[Projektové metriky] Sloiží k taktickým účelům, odhadují čas, náklady a monitorují projekt.
        \item[Metriky orientované na velikost] Odvozené z velikosti produktu a normalizovaná faktorem efektivity (např. počet řádků, počet stránek v dokumentaci).
        \item[Funkčně orientované metriky] Měří vztah mezi využitelností informační domény a složitostí systému (např. počet databázových dotazů určuje nutný výkon serveru).
      \end{description}

      Na projektu lze měřit čas, snahu, kvalitu a rozsah. U softwaru lze pak měřit počet defektů, produktivita a efektivita testů.

      \subsection{Kategorie}
        \begin{itemize}
          \item Čas - z evidence práce.
          \item Velikost - počet obrazovek, řádků kód nebo tříd.
          \item Pracnost - člověkohodiny.
          \item Kvalita - počet chyb v issue trackeru.
        \end{itemize}
