\section{Přednáška 3 -- Big Data}

\begin{description}
    \item[Big Data] \textit{(neformálně)} Objemná, rychle roustoucí data s (možnou) velkou rozmanitostí, vyžadující nové efektivní způsoby zpracování umožňující pokročilé rozhodování a nahlížení na ně.
    Pocházejí ze sociálních sítí, vědeckých výzkumů, senzorů, telefonů, \dots.
\end{description}

\subsection*{Charakteristiky (4V)}

\begin{description}
    \item[Volume] Velké množství dat rostoucí exponenciálně,
    \item[Variety] různé formy dat (xml, json, videa, příspěvky, senzory, \dots),
    \item[Velocity] zpracování a analýza velkého množství dat v reálném čase,
    \item[Veracity] data jsou neúplná, nekonzistentní a nejednoznačná.
\end{description}

\subsection*{Zpracování}

\begin{description}
    \item[OLTP] Online Transaction Processing. Typicky mnoho krátkých transakcí, které mají být rychlé. Efektivitu měříme podle počtu transakcí za vteřinu.
    \item[OLAP] Online Analytical Processing. Zejména select dotazy (agregace), nemusí být nutně rychlé
    \item[RTAP] \textbf{Real-Time Analytical Processing}.
\end{description}

\subsection*{Předpoklady dat}

\begin{itemize}
    \item Formát dat je neznámý a nekonzistentní,
    \item čtení dat převyšuje zápis,
    \item aktualizace dat není běžná (data se nahrazují).
\end{itemize}

Pro zpracování jsou potřeba nové techniky a technologie.
Využívají se distribuované systémy, \textit{MapReduce} a jiná programovací paradigmata, \textit{NoSQL} databáze, datové sklady a strojové učení.

Snaha napodobit povahu dat z reálného světa.

\subsection*{NoSQL}

Databáze nové generace cílící na použití, pro která relační databáze nejsou vhodné.
Klíčové vlastnosti jsou: nerelační, distrubuované, a horizonálně škálovatelné.
Často jsou charakterizované jako: schema-free, eventualy consistent.

\subsection*{Typy NoSQL}

\subsubsection*{Key-value}

Uložiště typu klíč-hodnota (redis, memcached).
Hodnota je \textit{blackbox} pro databázový stroj.
Vhodné pro jednoduchá data vyhledávána pouze klíčem,
    
\subsubsection*{Wide column}

Sloupcově orientovaná uložiště, kde řádky mají podobnou (ne nutně stejnou) strukturu.
Podobné relačním databázím, s proměnlivým počtem sloupců.
Využívá se \textit{column family} (tabulka), \textit{row} a \textit{column} (cassandra, HBase, Google Bigtable).
Vhodné pro logy, CMS a blogy (obecně cokoliv se strukturovanými daty).

\subsubsection*{Document stores}

Dokumentová uložiště hierarchických struktur (json, xml, \dots) identifikovaných klíčem.
Možné sestavovat složité dotazy.
Vhodné pro CMS, analytiky, logy, eshopy, dots (MongoDB, Couchbase).

\subsubsection*{Graph databases}

Orientované nebo neorientované grafy s uzly (node -- entity) a hranami (relationships).
Hrany i uzly můžou mít dodatečné vlastnosti.
Dotazy lze dělat pomocí grafových algoritmů, na podgrafy i nadgrafy.
Např. Neo4j.
Vhodné pro chemické sloučeniny, hledání cest, sociální sítě.


\subsection*{Výhody}

\begin{itemize}
    \item Škálování -- horizontální distribuce.
    \item Objemnost -- pojme i data nemožná zpracovat relační db.
    \item Administrování -- automatická údržba.
    \item Ekonomické -- možné využít levný HW.
    \item Flexibilní -- jednodušší změny díky absenci schematu.
\end{itemize}

\subsection*{Výzvy / nevýhody}

\begin{itemize}
    \item Maturity -- často v pre-produkční fázi (bez key-features).
    \item Podpora -- chybí z důvodu údržby jako OS.
    \item Analytiky -- chybějící byzsnys logika pro okamžité dotazování.
    \item Expertise -- chybějící odborníci.
\end{itemize}