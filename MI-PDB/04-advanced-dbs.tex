\section{Přednáška 4 -- Pokročilé databázové systémy}

\subsection*{Škálovatelnost}

Schopnost systému zpracovat rostoucí množství dat a/nebo dotazů bez ztráty výkonu.

\subsubsection*{Vertikální}

Škálování v rámci jednoho uzlu systému.
Jedná se o přidávání zdrojů (CPU, paměť, \dots).
Často volená varianta (jednoduchá na implementaci, bez problémů s distribucí).

Je ale dražší a má horní limit.
Může nastat vendor-lock (existuje jen pár výrobců).
Při nasazení je navíc nevyhnutelný výpadek systému.

\subsubsection*{Horizontální}

Přidávání nových uzlů do systému.
Častá volba při implementaci NoSQL.

Je levnější a flexibilnější, navíc bez \uv{úzkého hrdla}.

Oproti standardnímu řešení je ale složitější a navíc představuje nové problémy (distribuce, nekonzistence, synchronizace).
Při implementaci navíc často používáme falešné předpoklady (stabilní síť bez zpoždění, neomezený přenos, bezpečnost, neúdržbovost, \dots).

\subsection*{Modely distribuce}

Rozlišujeme dva typy distribuce - \textit{sharding} a \textit{replikace}.
Ty jsou navzájem nezávislé -- lze je použít zvlášť nebo nakombinovat.

\subsubsection*{Sharding}

Různá data na různých uzlech -- zvyšuje možnost pojmout více dat a zvyšuje výkon.
Data, která jsou často načítána společně by měla být uložena společně.

Snaha dosáhnout rovnoměrné distribuce dat a rozložení zátěže.
Tyto požadavky ale můžou často jít proti sobě nebo se časem měnit.

Rozlišujeme více strategií jak data rozmístit na uzly:

\begin{description}
    \item[Mapping structures] Data rozmístěna náhodně pomocí nějakého algoritmu.
    Informace o mapování musí být někde centrálně udržována.
    \item[General rules] Každý uzel má nějaká specifická data (hashování, rozdělování podle rozsahu).
\end{description}

Sharding je složitý, protože musíme rozhodovat jaký uzel použít nejen při ukládání, ale i čtení.
Ne každý uzel ale má znalost celého clusteru, navíc mohou uzly nebo jejich komunikace selhávat.

\subsubsection*{Replikace}

Stejná data na různých uzlech - zvyšuje výkon a snižuje možnost výskytu chyb.
Počet kopií je určený pomocí \textit{replikačního faktoru}.

Máme dva druhy architektur: \textit{master-slave} a \textit{peer-to-peer}.

Při použití \textbf{master-slave} máme jeden určený node, který je master a ten se stará o veškerou správu všech ostatních.
Čtení pak mohou zvládnout i slave uzly.

Pro zápis se používá pouze master, který změny propaguje.
Nastává ale problém s konzistencí -- propagace trvá nějaký čas.

Při použití \textbf{peer-to-peer} mají všechny uzly stejnou roli a zodpovědnosti.
Čtení i zápis může provést kdokoliv, čímž může vzniknout nekonzistence -- uzly se musí synchronizovat.

Architektury můžeme kombinovat, při použití sharding a master-slave, využijeme více masterů (každý pro jiná data) a jejich role se mohou překrývat se slave pro jiná data.

Při použití peer-to-peer se sharing jsou pak zjednodušeně všechna data všude.

\subsection*{CAP teorém}

\textit{Consistency}, \textit{Availability} a \textit{Partition tolerance} -- nelze mít všechny vlastnosti u distribuovaného systému, je nutné zvolit maximálně dvě.

\begin{description}
    \item[Consistency] Čtení i zápis prováděný atomicky (po zápisu všichni vidí stejná data).
    \item[Availability] Každý požadavek na čtení nebo zápis musí vyústit v odpověď.
    \item[Partition tolerance] Systém musí fungovat i při rozdělení na více izolovaných skupin (ztráta komunikace). 
\end{description}

Při výběru vlastností máme možnosti:

\begin{description}
    \item[CA] Jednoduché dosáhnout ACID, platí pro každý jednouzlový systém, případně clustery.
    \item[CP] S použitím distribuovaných zámků.
    \item[AP] Využívá se BASE místo ACID (cassandra, CouchDB). 
\end{description}

\subsection*{ACID}

Volí konzistenci před dostupností, typický pro relační databáze.

Vlastnosti ACID:

\begin{description}
    \item[Atomicity] Částečné zpracování transakce není povoleno (vše nebo nic).
    \item[Consistency] Transakce převádí databázi z jednoho konzistentního stavu do jiného.
    \item[Isolation] Paralelní transakce nevidí necommitnuté změny.
    \item[Durability] Commitnuté změny musí být trvalé. 
\end{description}

\subsection*{BASE}

Volí dostupnost před konzistencí, typický pro NoSQL, je lépe škálovatelný než ACID.

Vlastnosti BASE:

\begin{description}
    \item[Basically Available] Systém funguje převážně bez chyb (nastalé chyby nejsou fatální).
    \item[Soft state] Stav je nedeterministický (mění se v čase).
    \item[Eventual Consistency] Dříve nebo později se systém dostane do konzistentního systému.  
\end{description}
