\section{Přednáška 6 -- Neo4j}

Open source grafová databáze s ACID transakcemi.
Dotazování funguje pomocí jazyka \texttt{Cypher}.

\subsection*{Model}

Jedná se orientovaný multigraf.
Uzly mají unikátní id, množinu \textit{labels} a množinu \textit{properties}.
Relace mají unikátní id, směr a mohou mít \textit{properties}.

\texttt{Property} uzlu a relace je \textit{key-value}, kde \textit{key} je \texttt{String} a \textit{value} je primitivní typ (nebo kolekce primitivních typů).

\subsection*{Traversal framework}

Umožňuje spustit dotazy na průchod grafem.
Průchody se přesně drží ceste, které odpovídají dotazu.

\subsection*{Cypher}

Deklarativní dotazovací jazyk, můžeme specifikovat vlastnosti uzlů a hran, případně délku cesty.
Jazyk je ASCII grafický (uzly v závorkách, hrany šipkami).

Dotazujeme se klíčovým slovem \texttt{MATCH}, dále následuje ASCII reprezentace grafu.

\subsubsection*{DML}

Používáme klíčová slova \texttt{CREATE}, \texttt{DELETE}, \texttt{SET}, dots.
