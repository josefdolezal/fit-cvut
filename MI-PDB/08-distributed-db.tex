\section{Přednáška 8 -- MapReduce, Apache Hadoop}

\subsection*{MapReduce}

Programovací model a implementace (od Google) umožňující paralelní a distribuované výpočty s vysokým výkonem.

Funguje pomocí distrubuce dat mezi více uzlů, ty je následně zpracovávají paralelě.
Funkce \texttt{Map} rozdělí vstupní data problému na podproblémy a vytvoří \textit{key-value} páry.
Páry se stejným klíčem jsou následně řešeny společně v jednom volání funkce \texttt{Reduce}.

Funkce \texttt{Reduce} přijmá podřešení úlohy a vyřeší celkový celkový problém.

\subsubsection*{Architektura}

Nad clusterem je \texttt{MapReduce} řešen architekturou \textit{master-slave}.
Master přijme problém a rozdělí ho mezi uzly aby provedly funkci \textit{Map}.

Uzly provedou funkci map a notifikují \textit{mastera}.
Ten určí další uzly, které budou provádět funkci \textit{reduce}.
Nové uzly si využádají páry \textit{key-value} a provedou \textit{reduce}.
Po provedení notifikují \textit{mastera}, který následně úlohu ukončí.

Slouží k úlohám jako řazení, seskupování, počítání, sčítání, agregace dat.

\subsection*{Apache Hadoop}

Distribuované uložiště a zpracování velkého množství dat na clusteru.
Implementuje distribuovaný \textit{file system} a model \textit{MapReduce}.

\subsubsection*{Hadoop File System}

Distribuovaný \textit{file system} zaměřený na škálovatelnost a odolnost proti chybám.
Cluster se může skládat z tísíců uzlů a být tak náchylný na chyby.
HDFS umí chyby detekovat a automaticky se z nich zotavit.

Postavený je na architektuře \textit{master-slave}. Master (\texttt{NameNode}) spravuje \textit{namespace} a souborové bloky (jejich mapování na fyzické bloky) a je vstupním bodem pro všechny operace.
Slave (\texttt{NameNode}) fyzicky ukládá bloky souborů a zpracovává IO požadavky.

Pro zajištění dostupnosti se využívá replikování dat.
To je typicky nastaveno na 3 repliky.

DataNode uchovává každý blok (repliku) v samostatném souboru.
Upozorňuje NameNode pomocí HeartBeat.
