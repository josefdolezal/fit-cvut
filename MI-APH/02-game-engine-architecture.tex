\section{Přednáška 2 -- Architektura herních enginů}

Rozlišujeme tři základní úrovně počítačovúch her.
Herní aplikaci, herní smyčku a herní systém.

\subsection*{Herní aplikace}

Stará se o vše spojené se spuštění, aplikace, načítá nastavení pro herní okno a spouští herní smyčku.
Na konci životní cyklu hru ukončuje.

\subsection*{Herní smyčka}

Optimálně by neměla zabrat více než 16.6ms (při 60 FPS).
Některé systémy (např. fyzika, AI) mohou vyžadovat i častějíší volání smyčky.

\img{game-loop.png}{Standardní herní smyčka}{game-loop}{0.25}

\begin{enumerate}
    \item Inicializace,
    \item Zpracování vstupů,
    \item (Update) Zpracování herní logiky,
    \item Smazání a vykreslení herích objektů na obrazovce,
    \item Krok 2 nebo krok 6,
    \item Ukončení hry.
\end{enumerate}

\subsubsection*{Update metoda}

\begin{description}
    \item[Fixní časový skok] Každý update posune herní čas o fixní hodnotu.
    \item[Variabilní časový skok] Každý update posune herní čas o takovou hodnotu, kolik času reálně uběhlo od posledního updatu. Je to ale nedeterministické a také nevhodné pro některé moduly (např. fyzika).
    \item[Adaptivní časový skok] Přepíná mezi fixním a variabulním skokem.
\end{description}

Během updatu mohou být některé objekty v nekonzistentním stavu.
Před a po by ale měly vždy být v konzistentním.

Některé objekty se mohou vyskytnout ve stavu, že jsou o jeden snímek pozadu oproti ostatním.
Lze vyřešit např. pomocí dávkových aktualizací nebo \textit{script execution order} v Unity.

\subsubsection*{Vícevláknová herní syčka}

\img{multi-thread-game-loop.png}{Vícevláknová herní syčka}{multi-thread-game-loop}{0.4}

\subsection*{Herní systém}

Součástí každé hry, aktualizován během herní smyčky.

\subsection*{Komponenty enginů}

\medskip

\subsubsection*{Hlavní komponenty}

\medskip

\begin{description}
    \item[Herní smyčka],
    \item[Scene Manager] Spravuje herní objekty,
    \item[Resource Manager] Spravuje herní aktiva,
    \item[Input Manager] Řidí externí vstupy (myš, klávesnice, \dots),
    \item[Memory Manager],
    \item[Rigidbody Engine] Detekce kolizí na základě událostí,
    \item[Physics Rngine] Řídí chování objektů na základě kolizí,
    \item[Render Engine] Vykresluje objekty pomocí GPU, je kontrolován \textit{Scene grafem},
    \item[Animation Engine] Řídí animace,
    \item[Scripting Engine] Využívá interpretované jazyky,
    \item[Multimedia Engine] Přehrává hudbu, videa, \dots  
\end{description}

\subsubsection*{Další komponenty}

\medskip

\begin{description}
    \item[GUI Components] Tlačítka, checkboxy, textová pole,
    \item[Level Editor] Umožňuje vytvářet prostředí pomocí aktiv,
    \item[Shading Engine] Vypočítává speciální efekty (stíny, \dots),
    \item[Networking Engine] Řídí meziuzlovou komunikaci,
    \item[AI Engine] Řídí interakci objektů 
\end{description}

\subsection*{Scene Graph}

Jedná se o základní strukturu každé interaktivní aplikace, udržuje data v hierarchii.
Jedná se o \textit{N-tree} strukturu, kde rodičovské uzly ovlivňují své potomky (rotace, měřítko, \dots).
Listy stromu reprezentují grafická primitiva (tvary, meshe, \dots).

\subsubsection*{Scene Manager}

Spravuje objekty ve scéně.
Pracuje s nimi pomocí posílání zpráv, může je vyhledávat pomocí tagů, názvů, lokace, \dots

\subsection*{Resource Manager}

Poskytuje přístup ke všem aktivám.
Tedy k materiálů, shaderům, animacím, texturám, levelům.
Ty jsou enginem převáděny do jiných formátů (např. XML).

Manager se stará o životní cyklus assetů -- načítá je do paměti když jsou potřeba a odstraňuje je pokud už nadále potřeba nejsou.
Pro rychlejší přístup je možné použít cache.

\subsubsection*{Audio Engine}

Stará se o úpravu zvukových stop (např. akustika, úprava vzdálenosti, ozvěny, \dots).

\subsection*{Input Manager}

Detekuje vstupy z různých zařízení.
Obvykle podporuje primitivní události jako stisk kláves, stisk myši nebo pohyb myší.

Existují i doplňující události jako např. gesta prsty (pinch, double-tap, swipe).
Ty ale nemusí být vždy podporována.

Engine může podporovat i game-specific události (cheaty, komba, \dots) nebo přemapování vstupů.
Hry mohou závislosti sledovat kontextově (ovládní chůze vs. ovládání auta).

\subsection*{Memory manager}

Defaultní manažer paměti (v C runtimu) není pro mnohé enginy dostačující/vhodný a tak si vytvářejí vlastní.
Ve standardním manažeru dochází k fragmentaci, což se snaží manažery řešit.

\subsubsection*{Poll-based allocator}

Alokuje paměť po blocích, každý o stejné velikosti -- netrpí fragmentací, ale entity musejí být všechny stejné velikosti.

\img{poll-based-allocator.png}{Poll-based allocator}{poll-based-allocator}{0.7}

\subsection*{Načítání prostředí světa}

Existuje více přístupů k načítání herního světa.

\begin{description}
    \item[Level loading] Načítání hry po kusech -- vyžaduje loading screen.
    \item[Air locks] Načítání větších bloků (se scénou) a menších s \textit{air lockem}. Jakmile se uživatel octne v části odkud se nemůže vrátit, je starý blok zahozen a načten nový. Např. ve hře Portal.
    \item[Worldstreams] Používá se v open-world hrách. Svět je rozdělen na regiony -- jakmile vejdu do regionu (B) a jsem dostatečně daleko na to abych neviděl předchozí (A), můžu ho uvolnit a načíst následující (C). Např. GTA, RDR2 <3.
\end{description}
