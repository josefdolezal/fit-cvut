\section{Přednáška 5 -- Komponentová architektura}

\subsection*{Komunikace}

Rozlišujeme více druhů komunikace mezi objekty ve hrách.

\subsubsection*{Modifikace stavu kontejneru}

Nepříma komunikace objektů, která je ale hůře debugovatelná.
Implementována např. sdílenou \textit{state machine}.

\subsubsection*{Přímá volání}

Standardní způsob komunikace objektů v OOP.
Je rychlý, ale zavádí úzké propojení objektů.
Např. skupina skupina objektů, které jsou propojeny vzájemně.

\subsubsection*{Zasílání zpráv}

Zprávy jsou zasílány jsou \textbf{eventy} a \textbf{commandy}.
Každá komponent definuje, o jaké zprávy má zájem.
Není tak rychlý jako přímá volání, ale cena/zpomalení je zanedbatelné všude kromě nejkritičtějíších částí kódu.
Lze využít např. pro broadcast informace o konci hry.

Je několik možností implementace.
Např. pomocí polymorfismu, anonymních funkcí, ukazatelů na funkce, \dots

Komponenty by měly být upozorněny na každou změnu stavu, která je pro ně významná.
Je možné použít také \textit{blind event forwarding}, kdy objekt přepošle event jinému objektu.

Aby bylo možné doručovat zprávy se zpožděním, zavádí se také fronta \textit{event queue}.

\subsubsection*{Typy zpráv}

\medskip

\begin{description}
    \item[Unicast] Jedna komponenta odesílá zprávu druhé napřímo. Většinou implementováno pomocí přímých volání.
    \item[Multicast] Rozesílání informace specifickým odběratelům nebo všem komponentám splňující nějakou podmínku.
    \item[Broadcast] Typicky pro komunikaci typu systém-entity (např. dokončení úrovně, \dots).
\end{description}
