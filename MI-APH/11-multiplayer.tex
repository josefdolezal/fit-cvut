\section*{Přednáška 11 -- Multiplayer}

Možnost více hráčů působit ve stejné hře ve stejném čase.

\subsection*{Typy multiplayerů}

\begin{description}
    \item[Single screen] Hráči spolu hrají v rámci jedné obrazovky.
    \item[Split screen] Uživatelé spolu hrají v rámci jedné obrazovky, ale každý se může nacházet v jiné části.
    \item[Networked] Multiplayer po síti.
    \item[MMOG] \textit{Massive multiplayer online games} Např. WoW
\end{description}

\subsubsection{Topologie}

Vyskytují se problémy jako synchronizace.
Všichni klienti musí dosáhnout alespoň nějaké úrovně synchronizace.
Typicky se přenáší minimální množství stavu potřebného k sestavení celkové informace.

\subsubsection*{Peer-to-peer}

Všechna zařízení si vyměňují data navzájem (úplný graf).
Více možností jak hru uspořádat (např. jeden hráč master, ostatní slave nebo pomocí zodpovědností -- každý za něco).

\subsubsection*{Client-server}

Pro \(n\) zařízení je potřeba \(n-1\) spojení.
Server se stává bottle-neck celé architektury.

\subsubsection*{Zasílání zpráv}

\begin{description}
    \item[Update] Typicky nepotřebují povtrzení, jako relevantní se bere poslední zpráva.
    \item[Command] Zprávy, které mají vliv na stav hry a musí být potvrzeny.
    \item[Action] Zprávy s vysokou prioritou (vstup uživatele, \dots).
    \item[Procedure call] Volání vzdálených funkcí pomocí jména a předaných atributů (např. přehrání zvuku).
\end{description}

\subsection*{Pojmy}

\begin{description}
    \item[Latence] Množství času mezi vyvoláním akce a jejím provedením. 
\end{description}

\subsection*{Synchronizace}

\subsubsection*{Extrapolace}

Klient funguje 60FPS, server 10-30FPS.
Jakmile klient dosáhne nového stavu, plynule do něj v průběhu času přejde.

\subsubsection*{Predikce}

Klient předpovídá budoucí hodnoty.

\subsubsection*{Server-side rewind}

Používaní na zásahy na dlouhou vzdálenost.
Server převede stav přesně do toho, jak vypadal v momentě kdy uživatel vykonal akci.

\subsection*{Bezpečnost}

\begin{description}
    \item[Map hacking] Odstraňování vlastnosti mapy pro účel nalezení skrytých vlastností.
    \item[Bot cheat] Bot, který buď hraje za uživatele nebo mu pomáhá. 
\end{description}
