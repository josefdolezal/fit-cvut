\section{Přednáška 6 -- Komponentová architektura, Návrhové vzory}

\subsection*{Dvoufázová inicializace}

Vyhýbá se inicializaci objektů jen pomocí konstruktoru.
Objekty na sobě mají např. metody \texttt{LoadFromJSON(:)} nebo \texttt{LoadFromXML(:)} na inicializaci.
Konstruktor zpravidla jen vytváří objekt, ten je inicializován pomocí inicializačních metod.

\subsection*{Flyweight pattern}

Používá se pro podporu typů, které požadují velké množství instancí.
Jeden objekt drží sdílený data všech objektů, které zastupuje a nabízí veřejné rozhraní, přes které je možné přistupovat ke konkrétním objektům.

\subsection*{Flag}

Pole bitů držící binární vlastnosti herního objektu.
Může být využito \textit{Scene Manager} k rychlému vyhledávání objektů s nějakou vlastností (všechny mrtvé, \dots).

Implementace je pomocí bitového pole, kde předem víme jaký index znamená jakou vlastnost (\textit{visible}, \textit{dead}, \dots).

\subsection*{Dirty flag}

Označuje objekt, který byl změněn -- musí být vždy nastaven.

\subsection*{Replay}

Umožňuje reprodukovat libovolný stav hry v libovolném čase.
Je náročnějíší než standardní ukládání hry, protože každá entita musí být reprodukovatelná.

Lze implementovat pomocí ukládání celého stavu \textbf{všech} herních objektů (po framu nebo s určitou frekvencí), reprodukce pak proběhne nastavením stavu všem objektům.

Druhá možnost je ukládat herní \textit{eventy} tak jak byly odesílány.
Při přehrávání se zablokuje možnost posílání zpráv a posílají se ty předešlé.
Funguje lépe pokud zprávy jsou posílány pomocí nějakého systému (žádné přímé zprávy).
