\section*{Přednáška 9 -- AI ve hrách}

\subsection*{Hledání cest}

Potřebné v každé hře (využívající mapu) k výpočtům pohybů a analýze prostředí.

\subsubsection*{Navigační graf}

Abstrakce všech lokací ve hře, které může agent navštívit.
Umožňuje agentům jednoduše vypočítat cesty, které se vyhýbají vodě, preferují cesty po silnici, \dots
Rozlišujeme více druhů grafu:

\begin{description}
    \item[Waypoint-based] Level designer umisťuje do mapy body, které jsou později spojeny.
    \item[Mesh-based] Vytvořen z polygoniální reprezentace země prostředí a popisuje přístupné oblasti.
    \item[Grid-based] Tvořen mřížkou nad herním prostředím. Dlaždice mřížky mají flag přístupnosti. Mřížka nemusí být pouze čtvercová, exisutují i hexagonální nebo \uv{osmisměrkové}. 
\end{description}

\subsection*{Algoritmy vyhledávání}

Cesty mohou být v grafu vyhledávány různými algoritmy.
Lze použít např. \texttt{DFS} (nezaručuje nejkratší cestu), \texttt{BFS} (zaručuje nejkraší) nebo pokročilé algoritmy beroucí v potaz i cenu průchodu.

\textit{Cost-based} algorimy berou v úvahu i cenu průchodu.
Dijkstrův algoritmus dokáže z jednoho uzlu najít nejkratší cestu do všech ostatních uzlů, provede při tom ale spoustu nepotřebných návštěv.

Vylepšením Dijkstrova algoritmu o přidání heuristické hodnoty k hodnotící funkci dostaneme algoritmu \texttt{A*}.
Heuristická hodnota je určena jako odhad ceny z každého uzlu do počátečního.
Jako heuristickou funkci lze použít Euklidovskou vzdálenost, Manhattanskou, adaptivní (penalizace a změnu směru), \dots

BFS ignoruje cenu, Dijkstra naopak počáteční pozici.
A* bere v úvahu oboje.

\subsection*{Základní techniky AI}

\subsubsection*{Skriptování}

Natvrdo zadrátované chování umělé inteligence.
Jednoduché a snadno debugovatelné, ale aby bylo kvalitní musí pokrýt velké množství možností.

\subsubsection*{Finite state machine (FSM}

Nejstarší a nejběžnější formalizace AI.
Jedná se o čtveřici \((Q, \sigma, \delta, q_0)\) kde \(Q\) je neprázdná konečná množina stavů, \(\sigma\) je neprázdná množina vstupů, \(\delta\) je přechodová funkce a \(q_0\) je počáteční stav.

Každá entita je vždy v právě jednom stavu.

Existuje i hierarchická verze, kde každý stav může mít i svůj nadstav a podstav.
Stavy lze seskupovat, sdílejí pak přechody do jiných skupin.

\subsubsection*{Rozhodovací stromy}

Strom hierarchických uzlů řídících rozhodovací proces.
Vnitřní uzly vedou k patřičným listům vhodným v dané situaci.

Rozlišujeme více druhů těchto stromů.

\begin{description}
    \item[Selector] Zkouší vyhodnocovat své potomky zleva doprava dokud nedostane pozitivní odpověď.
    \item[Sequence] Zkouší potomky spouště metodou \uv{if all succeed}.
    \item[Decorator] Umožňuje dynamické přidávání chování bez modifikace kódu. Každý uzel má je jednoho následníka.
    \item[Parallel] Paralelně vyhodnocuje následníky.
    \item[Action] Uzel je využit k interakci s herním prostředím, může být reprezentován skriptem, atomickou akcí nebo dalším BT.
    \item[Condition] Uzel sleduje herní prostředí a odpovídá buď úspěchem nebo selháním. Buď okamžité nebo průběžné. 
\end{description}

\subsection*{Pokročilé AI}

\begin{description}
    \item[Bot] Hráč ovládaní pomocí AI simulující skutečného hráče.
    \item[Planning] Formalizovaný proces hledání posloupnosti akcí k naplnění cíle.
    \item[Supervised learning] Jako standardní učení. Pro určité vstupy existuje správný výstup, který se algoritmus musí naučit.
    \item[Unsupervised learning] U výstupu nelze rozhodnout jestli je správný nebo, alrgoritmus se místo toho učí vzory.
    \item[Reinforcement learning] Agent se učí nejlepší nejlepší možný výsledek pomocí metody pokus-omyl. Na každou akci dostává zpětnou vazbu.
\end{description}

\subsubsection*{Plánování akcí}

\begin{description}
    \item[GOAL] Zaměřuje se na myšlenku popisu cílů pomocí dosažitelných stavů okolního stavu. Např. co musím udělat, pokud chci útočit (vzít zbraň, najít úkryt, nabýt, zaútočit). Každý ze stavů se zobrazuje v rozhodovacím stromu.
\end{description}

\subsubsection*{Realm-time strategy AI}

AI strategií rozdělujeme do tří vrstev (z hlediska architektury):

\begin{description}
    \item[Micromanagement] Kooperace, optimalize a reakce.
    \item[Tactics] Časování, postup cílení.
    \item[Strategy] Technologie, jednoty (armádní), ekonomika.
\end{description}

\subsubsection*{Case-based uvažování}

Při rozhodování zohledňujeme rozhodnutí z minulosti.
Pokud nějaká situace dříve nenastala, vybere podobnou událost a její výstup pvrizpůsobí aby odpovídal aktuální situaci.
