\documentclass[11pt,a4paper,czech]{article}

\usepackage[utf8]{inputenc}
\usepackage[IL2]{fontenc}
\usepackage{a4wide}

% Page heading
\newcommand{\heading}[1]{
    \begin{center}
    {\bf\large #1}\par
    \bigskip
    \end{center}
}

\begin{document}
\heading{FI-KSA - Refrelexe na přednášku Etika}

Pro svou reflexi jsem si vybral poslední z přednášek s tématem Etika.
V první části přednášky jsme se věnovali problematice etiky antropologů.
Tedy jakým způsobem se by se měli během svého výzkumu chovat, aby měl co nejmenší dopad na okolí, ve kterém je prováděn.

Druhá část přednášky se zabývala etikou ve společnosti.
Náplní bylo seznámení se zajímavými sociálními experimenty a také uvědomění si, z jakých důvodů se občas chováme způsobem, který bychom jindy odsoudili.

\medskip

Rozhodnutí napsat reflexi na toto téma jsem učinil hned na první přednášce.
Důvodem pro mě byly okolnosti, s kterými jsem se v té době potýkal ve své práci.

Před začátkem šestého semestru jsem se rozhodl opustit stávající práci a věnovat se plně bakalářské práci.
Své rozhodnutí jsem oznámil v průběhu ledna, práci jsem opouštěl koncem března.

K mému překvapení byly následující týdny tak vyostřené, že jen vzájemný pozdrav s nadřízenými se stal takřka nepřekonatelným problémem.
Paradoxně se situace zhruba dva dny před první přednáškou vyhrotila tak, že už jsem na poslední pracovní dny ani nedocházel do společných kanceláří.

Na této přednášce jsem se následně dozvěděl, že během semestru se bude toto téma probírat.
Tento týden jsem se tedy konečně dočkal.

Při zpětném pohledu na pracovní situaci si myslím, že jde uplatnit ponaučení z Milgramova experimentu, o kterém jsem se na přednášce dozvěděl.
Průběh byl alespoň z mé strany úplně stejný.
Během posledních pracovních dní jsem si s nadřízenými vyměnil řadu nepřívětivých pohledů, nevlídných slov i vyhraněných názorů.

V tom přesně vidím analogii s elektrickými spínači Milgramova experimentu.
Přestože každé jedno gesto samo o sobě nebylo velkým prohřeškem proti morálním zásadám, jejich častá výměna vedla v konečném důsledku k velmi nepříjemným situacím.
Věřím, že kdybych tehdy dokázal na situaci nahlížet tak, jak bych na ní nahlížel dnes, s vědomostmi z poslední přednášky, postavil bych se k celému problému jinak a lépe.
Ne nadarmo se jistě říká, že moudřejší ustoupí.

\medskip

Poslední přednáška měla navíc velmi příjemnou atmosféru.
Za celých šest semestrů byla první, kterou jsem absolvoval mimo sklepní prostor či místnost (téměř) bez oken.
Na přednáškách tohoto předmětu bych pravděpodobně neměnil vůbec nic.
Každá byla velmi zajímavá a dalo se při ní příjemně odpočinout od technologických témat, která nás jindy provází od rána do noci.

\vspace{1.5cm}
\hfill{}Josef Doležal
% Remove page numbering
\thispagestyle{empty}
\end{document}