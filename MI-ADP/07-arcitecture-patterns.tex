\section{Přednáška 7 -- Architektonické vzory}

\subsection{Monolitické aplikace}

Aplikace, ve kterých se různé funkcionality (vstupy, výstupy, zpracování chyb, UI, \dots) úzce prolínají místo toho, aby byly oddělené do komponent.
Nabízejí rychlý vývoj a nasazení, mohou nastat problémy při škálování a údržbě.

\subsection{Component-based}

Rozdělování aplikace do více komponent, které by měly být jednoduše nahraditelné.
Splňuje SoC.
Dosáhnout této architektury je možné více způsoby:

\begin{itemize}
    \item V rámci jedné binárky -- knihovny, moduly, balíčky, třídy, objekty
    \item V rámci služeb -- Webové služby, microservice
\end{itemize}

\subsubsection{Dependency injection}

Umožňuje vložit závislosti klientskému objektu tak, že klient nemusí vědět jak konkrétně objekt vyrobit.
To umožňuje měnit implementaci závislostí bez nutnosti úpravy klienta, je např. možné měnit konfiguraci.

\subsection{Layered}

Návrh s několika úrovněmi abstrakce, které oddělují jednotlivé zodpovědnosti do vrstev.
Každá vrstva reprezentuje různý model stejné informace s lišící se znalostí detailu (např. ISO/OSI).
V běžných aplikacích se odděluje např. prezentační, logická (business) a datová vrstva.
