\section{Přednáška 6 -- Behavioral Návrhové vzory}

\subsubsection{Chain of Responsibility}

Metoda předávání požadavků zřetězenými objekty.
Při vyřizování požadavku je zavolána obslužná funkce, která po svém dokončení volá další obslužnou funkci.

\subsubsection{Template method}

Nadtřída definuje kostru algoritmu pomocí několika kroků.
Podtřídy reimplementací některých kroků mohou měnit chování algoritmu.
Například použito v \texttt{UIKit} o životního cyklu \texttt{Controlleru}.

\subsubsection{Mediator}

Zapouzdřuje interakci množiny objektů.
Např. pro dialog výběru fontu mohu mít \texttt{FontDialogDirector}, který obsahuje seznam fontů a textové pole.
Při změně výběru v seznamu je notifikován \texttt{Mediator}, který na základě výběru aktualizuje textové pole.

\subsubsection{Iterator}

Umožňuje sekvenční přístup ke kontejneru bez nutnosti odhalení vnitřní reprezentace.
Obsahuje \texttt{Aggregate} (objekt vytvářející iterátor -- kolekce) a \texttt{Iterator} (objekt iterující nad kolekcí -- typicky obsahuje metody jako \texttt{first()}, \texttt{next()}, \dots).

\subsubsection{Mock}

Napodobuje chování reálných objektů.
Pomocí nich je možné např. během testování simulovat situace a stavy, do nichž se objekty dostanou v reálném světě ale typicky ne během testování.
