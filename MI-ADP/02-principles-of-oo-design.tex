\section{Přednáška 2 -- Principy OO návrhu}

K vyhnutí se špatnému návrhu aplikace je doporučené řídit se známými principy.
Sadu těchto principů popsali GoF a jsou zmíněné níže.

\subsection{DRY -- Do not repeat yorself}

Každá část aplikace musí mít jedinečnou, nezaměnitelnou a spolehlivou reprezentaci v rámci systému.

\subsection{KISS -- Keep it simple, stupid}

Při implementaci je důležité dbát na jednoduchost.
Složité systémy pracují méně spolehlivě a obsahují zbytečné prvky.

\subsection{SoC -- Separation of Concerns}

Rozdělení programu na patřičné části tak, aby každá měla zodpovědnost pouze za oddělenou funkcionalitu.
Takto vytvořené programy jsou více modulární.

\subsection{YAGNI -- You are not going to need it}

Do programu by neměla být přidávána funkcionalita, která není nezbytně nutná (netýká se vlastností, které umožňují další rozšířitelnost).

\subsection{Demeterův zákon}

Každá komponenta by měla mít znalost pouze nejbližší vlastností jiných komponent -- \texttt{a.b.c()} vs. \texttt{a.c()}.

\subsection{ZOI -- Zero, One, Infinity}

Objekty by měli mít své \texttt{property} pouze v počtech žádný, jeden nebo kolekce.
Jiné počty jsou nepřípustné.

\subsection{SOLID principy}

\begin{description}
    \item[SRP -- Single responsibility principle] Každý objekt by měl být zodpovědný pouze za jednu věc.
    \item[OCP -- Open-Closed principle] Entity jsou otevřené rozšíření ale uzavřené úpravě -- přidání funkcionality je možné bez úpravy stávajících.
    \item[LSP -- Liskov Substitution principle] Podtypy musí být nahraditelné nadtypem strukturálně i funkcionálně -- podtypy nekladou větší omezení na své funkce.
    \item[ISP -- Interface Segregation principle] Klienti by neměli být nuceni záviset na funkcích, které nevyužívají (interface by měl být rozdělen).
    \item[DIP -- Dependency Inversion principle] Moduly by neměli záviset na konkrétní implementaci ale pouze na rozhraní.
\end{description}
