\section{Přednáška 3 -- Klasifikace OO vzorů}

\subsection{Architektura vs. Vzor}

Architektury řeší problémy distribuované funkcionality, modularity, rozhraní, škálovatelnostim,\dots
Architektura komponenty je definování jejího chování v rámci celého systému.

Návrhové vzory jsou jazykově nezávislé abstrakce s využitím OOP modelované pomocí UML.

\subsection{Klasifikace OO vzorů}

Vzory se dělí podle toho co dělají (\textit{Purpose}) a kde se aplikují (\textit{Scope}).

\subsubsection{Purpose}

\begin{description}
    \item[Creational] Starají se o vytváření nových objektů
    \item[Structural] Starají se o kompozici tříd a objektů
    \item[Behavioral] Starají se o interakce mezi objekty a třídami
\end{description}

\subsubsection{Scope}

\begin{description}
    \item[Class patterns] Zaměřují se na vztahy tříd a podtříd (dědičnost)
    \item[Object patterns] Zaměřují se na vztahy mezi objekty (kompozice)
\end{description}

\begin{tabular}{ | l || p{3.5cm} | p{3.5cm} | p{4.2cm} | }
    \hline
    Scope / Purpose & Creational & Structural & Behavioral \\ \hline \hline
    Class & Factory & Adapter & Interpreter, Template \\ \hline
    Object & Abstract factory, Builder, prototype & Adapter, Bridge, Favade, Proxy & Command, Iterator, Mediator, Memento, State, Strategy \\ \hline
\end{tabular}


\subsection{Architektura MVC}

Architektura rozdělující aplikaci do tří vrstev: \texttt{Model}, \texttt{View} a \texttt{Controller}.

\subsection{Architektura MVP}

Obdobné jako MVC, využívá se u desktopových/mobilních aplikací.
Více \uv{flat} diagram oproti MVC.

\subsection{Architektur MVVM}

Odstraňuje kompletně logiku z vrstvy \texttt{View}, částečně řeší testovatelnost GUI aplikací.

\subsection{Návrhové vzory}

\subsubsection{Observer}

Notifikuje závislé objekty na změnu stavu.
Obsahuje \texttt{Subject} a \texttt{Observer}.

\subsubsection{Factory}

Vytváření objetků porušuje DIP, zavádí se proto factory třídy, které mají vytváření objetků na starosti.
Typicky vytváří objekty, které implementují společné rozhranní -- klientovi nezáleží na implementaci.

\subsubsection{Strategy}

Definuje rodinu algoritmů, které mohou být navzájem zaměnitelné.
Obsahuje \texttt{Context}, interface \texttt{Strategy} a následně konkétní implementace.

\subsubsection{State}

Umožňuje objektu zaměnit své chování na základě vnitřńiho stavu.
Obsahuje \texttt{Context}, interface \texttt{State} a konkrétní implementace.

\subsubsection{Visitor}

Umožňuje přidat nové operace do hierarchie tříd bez jejich změny.
Obsahuje \texttt{Visitor} (obsahující metodu \texttt{visit(\_:)}) a \texttt{Element} (obsahující metodu \texttt{accept(\_:)}).

\subsubsection{Proxy}

Poskytuje náhradu za jiný objekt, k němuž kontroluje přístup -- typicky implementuje stejné rozhranní.
Lze využít např. při logování.

\subsubsection{Command}

Zapouzdřuje zprávu objektu do samostatného objektu.
Obsahuje \texttt{Invoker} (volá \texttt{execute(\_:)}), \texttt{Receiver} (na kom je příkaz vykonán) a samotný \texttt{Command} interface.

\subsubsection{Memento}

Snapshot interního stavu objektu, který může být později obnoven.
Obsahuje \texttt{Originator} (vytváří memento se svým stavem), \texttt{Caretaker} (stará se o uchování mementa), \texttt{Memento} (samotný vnitřní stav).
